\documentclass[10pt, letterpaper]{article}

% Packages:
\usepackage[
    ignoreheadfoot, % set margins without considering header and footer
    top=2 cm, % seperation between body and page edge from the top
    bottom=2 cm, % seperation between body and page edge from the bottom
    left=2 cm, % seperation between body and page edge from the left
    right=2 cm, % seperation between body and page edge from the right
    footskip=1.0 cm, % seperation between body and footer
    % showframe % for debugging 
]{geometry} % for adjusting page geometry
\usepackage{titlesec} % for customizing section titles
\usepackage{tabularx} % for making tables with fixed width columns
\usepackage{array} % tabularx requires this
\usepackage[dvipsnames]{xcolor} % for coloring text
\definecolor{primaryColor}{RGB}{0, 0, 0} % define primary color
\usepackage{enumitem} % for customizing lists
\usepackage{fontawesome5} % for using icons
\usepackage{amsmath} % for math
\usepackage[
    pdftitle={Alisa Jung's CV},
    pdfauthor={Alisa Jung},
    colorlinks=true,
    urlcolor=primaryColor
]{hyperref} % for links, metadata and bookmarks
\usepackage[pscoord]{eso-pic} % for floating text on the page
\usepackage{calc} % for calculating lengths
\usepackage{bookmark} % for bookmarks
\usepackage{lastpage} % for getting the total number of pages
\usepackage{changepage} % for one column entries (adjustwidth environment)
\usepackage{paracol} % for two and three column entries
\usepackage{ifthen} % for conditional statements
\usepackage{needspace} % for avoiding page brake right after the section title
\usepackage{iftex} % check if engine is pdflatex, xetex or luatex

% Ensure that generate pdf is machine readable/ATS parsable:
\ifPDFTeX
    \input{glyphtounicode}
    \pdfgentounicode=1
    \usepackage[T1]{fontenc}
    \usepackage[utf8]{inputenc}
    \usepackage{lmodern}
\fi

\usepackage{charter}

% Some settings:
\raggedright
\AtBeginEnvironment{adjustwidth}{\partopsep0pt} % remove space before adjustwidth environment
\pagestyle{empty} % no header or footer
\setcounter{secnumdepth}{0} % no section numbering
\setlength{\parindent}{0pt} % no indentation
\setlength{\topskip}{0pt} % no top skip
\setlength{\columnsep}{0.15cm} % set column seperation
\pagenumbering{gobble} % no page numbering

\titleformat{\section}{\needspace{4\baselineskip}\bfseries\large}{}{0pt}{}[\vspace{1pt}\titlerule]

\titlespacing{\section}{
    % left space:
    -1pt
}{
    % top space:
    0.3 cm
}{
    % bottom space:
    0.2 cm
} % section title spacing

\renewcommand\labelitemi{$\vcenter{\hbox{\small$\bullet$}}$} % custom bullet points
\newenvironment{highlights}{
    \begin{itemize}[
        topsep=0.10 cm,
        parsep=0.10 cm,
        partopsep=0pt,
        itemsep=0pt,
        leftmargin=0 cm + 10pt
    ]
}{
    \end{itemize}
} % new environment for highlights


\newenvironment{highlightsforbulletentries}{
    \begin{itemize}[
        topsep=0.10 cm,
        parsep=0.10 cm,
        partopsep=0pt,
        itemsep=0pt,
        leftmargin=10pt
    ]
}{
    \end{itemize}
} % new environment for highlights for bullet entries

\newenvironment{onecolentry}{
    \begin{adjustwidth}{
        0 cm + 0.00001 cm
    }{
        0 cm + 0.00001 cm
    }
}{
    \end{adjustwidth}
} % new environment for one column entries

\newenvironment{twocolentry}[2][]{
    \onecolentry
    \def\secondColumn{#2}
    \setcolumnwidth{\fill, 4.5 cm}
    \begin{paracol}{2}
}{
    \switchcolumn \raggedleft \secondColumn
    \end{paracol}
    \endonecolentry
} % new environment for two column entries

\newenvironment{threecolentry}[3][]{
    \onecolentry
    \def\thirdColumn{#3}
    \setcolumnwidth{, \fill, 4.5 cm}
    \begin{paracol}{3}
    {\raggedright #2} \switchcolumn
}{
    \switchcolumn \raggedleft \thirdColumn
    \end{paracol}
    \endonecolentry
} % new environment for three column entries

\newenvironment{header}{
    \setlength{\topsep}{0pt}\par\kern\topsep\centering\linespread{1.5}
}{
    \par\kern\topsep
} % new environment for the header

\newcommand{\placelastupdatedtext}{% \placetextbox{<horizontal pos>}{<vertical pos>}{<stuff>}
  \AddToShipoutPictureFG*{% Add <stuff> to current page foreground
    \put(
        \LenToUnit{\paperwidth-2 cm-0 cm+0.05cm},
        \LenToUnit{\paperheight-1.0 cm}
    ){\vtop{{\null}\makebox[0pt][c]{
        \small\color{gray}\textit{Last updated in July 2024}\hspace{\widthof{Last updated in July 2024}}
    }}}%
  }%
}%

% save the original href command in a new command:
\let\hrefWithoutArrow\href

% new command for external links:


\begin{document}
    \newcommand{\AND}{\unskip
        \cleaders\copy\ANDbox\hskip\wd\ANDbox
        \ignorespaces
    }
    \newsavebox\ANDbox
    \sbox\ANDbox{$|$}

    \begin{header}
        \fontsize{25 pt}{25 pt}\selectfont Alisa Jung

        \vspace{5 pt}

        \normalsize
        \mbox{\hrefWithoutArrow{https://alisajung.github.io/}{alisajung.github.io}}%
        %\kern 5.0 pt%
        %\AND%
        %\kern 5.0 pt%
        %\mbox{\hrefWithoutArrow{https://www.linkedin.com/in/alisa-jung-25b37ab5/}{linkedin.com/in/alisa-jung-25b37ab5/}}%
        %\kern 5.0 pt%
        %\AND%
        %\kern 5.0 pt%
        %\mbox{\hrefWithoutArrow{https://github.com/yourusername}{github.com/yourusername}}%
    \end{header}

    \vspace{5 pt - 0.3 cm}


    
%    \section{Quick Guide}

 %   \begin{onecolentry}
  %      \begin{highlightsforbulletentries}


   %     \item Each section title is arbitrary, and each section contains a list of entries.

    %    \item There are 7 unique entry types: \textit{BulletEntry}, \textit{TextEntry}, \textit{EducationEntry}, \textit{ExperienceEntry}, \textit{NormalEntry}, \textit{PublicationEntry}, and \textit{OneLineEntry}.

     %   \item Select a section title, pick an entry type, and start writing your section!

      %  \item \href{https://docs.rendercv.com/user_guide/}{Here}, you can find a comprehensive user guide for RenderCV.


       % \end{highlightsforbulletentries}
    %\end{onecolentry}

    \newcommand{\sectionspace}{\vspace{.45cm}}

%\sectionspace
\vspace{.5cm}
    \section{Education} 

    \newcommand{\expspace}{\vspace{.25cm}}
%\newcommand{\educspace}{\vspace{.4cm}} 
\newcommand{\educspace}{\expspace} 
\newcommand{\sectspacetop}{\vspace{.1cm}}
\sectspacetop
        \begin{twocolentry}{
            Aug 2017 - May 2024
        }
            \textbf{Karlsruhe Institute of Technology}, PhD in Computer Science (summa cum laude)\end{twocolentry}

        \vspace{0.10 cm}
        \begin{onecolentry}
            \begin{highlights}
            \item \textbf{Thesis:} Mollifying Realistic Image Synthesis for Time Constrained Rendering
                \item \textbf{Research:} Physically based and spectral rendering, fluorescence, regularization, path guiding.
                \item \textbf{Programming}: Mostly C, C++, Python, some bash. Linux and Windows.
                \item \textbf{Teaching:} Exercise for lectures, advising student projects and theses.
            \end{highlights}
        \end{onecolentry}

        \educspace

        \begin{twocolentry}{
            Apr 2015 - Jun 2017
        }
            \textbf{Karlsruhe Institute of Technology}, M.Sc. in Computer Science\end{twocolentry}

        \vspace{0.10 cm}
        \begin{onecolentry}
            \begin{highlights}
            \item \textbf{Thesis:} Fluorescence in Bidirectional Rendering
            \end{highlights}
        \end{onecolentry}

        \educspace   
        
        \begin{twocolentry}{
            Oct 2016 - Mar 2017
        }
            \textbf{Cornell University,} Ithaca NY. Semester abroad for Master's thesis\end{twocolentry}

        \educspace

        \begin{twocolentry}{
            Sep 2011 - Mar 2015
        }
            \textbf{Karlsruhe Institute of Technology}, B.Sc. in Computer Science\end{twocolentry}

        \vspace{0.10 cm}
        \begin{onecolentry}
            \begin{highlights}
            \item \textbf{Thesis:} Irradiance Importance Sampling
            \end{highlights}
        \end{onecolentry}

    
    \sectionspace
    \section{Experience}
      \sectspacetop

      \begin{twocolentry}{
            Since Feb 2025
        }
            \textbf{Software Engineer}, CAS Software AG, Karlsruhe\end{twocolentry}

        \vspace{0.10 cm}
        \begin{onecolentry}
            \begin{highlights}
                \item Workflow automation, web services (Java, Python, C\#)
            \end{highlights}
        \end{onecolentry}

      \expspace
      
        \begin{twocolentry}{
            Jan 2023 – Jun 2023
        }
            \textbf{Visiting Rendering Researcher}, Weta Digital / Unity -- Wellington, New Zealand\end{twocolentry}

        \vspace{0.10 cm}
        \begin{onecolentry}
            \begin{highlights}
                \item Physically based rendering, path guiding and regularization in Manuka (C++)
            \end{highlights}
        \end{onecolentry}

\expspace

        \begin{twocolentry}{
            May 2015 – Jul 2015
        }
            \textbf{Lecturer}, Duale Hochschule Baden-Württemberg -- Karlsruhe\end{twocolentry}

        \vspace{0.10 cm}
        \begin{onecolentry}
            \begin{highlights}
                \item Lecturer for "Mobile Application Development"
            \end{highlights}
        \end{onecolentry}
        
        \expspace
             
        \begin{twocolentry}{Oct 2012 - Sep 2016}
        \textbf{Student Assistant}, Karlsruhe Institute of Technology / Fraunhofer IOSB        
        \end{twocolentry}
        \vspace{0.10 cm}
        \begin{onecolentry}
        \begin{itemize}
              \item Programmer (C++) for data-driven BRDFs in photorealistic rendering (summer 2016) %Mar 2016 - Sep 2016
                \item Programmer (C++) for path planning for a mobile robot platform and arm (summer 2014)%Mar 2014 - Sep 2014
                \item Programmer (Java) for distributed smart home applications (summer 2013) %Jun 2013 - Sep 2013
                \item Teaching Assistant tutoring for "Basic notions of computer science" (each winter term)
                \end{itemize}
        \end{onecolentry}


        
        

        %\vspace{0.10 cm}
        %\begin{onecolentry}
        %    \begin{highlights}
        %        \item Event planning, various roles in dorm self-government
        %    \end{highlights}
        %\end{onecolentry}

    \sectionspace
    \section{Voluntary Activities}
    \sectspacetop
    \begin{twocolentry}{
            Since 2018
        }
            \textbf{Ultimate Frisbee Coach}, MTV Karlsruhe
           \end{twocolentry}
        \begin{onecolentry}
            \begin{highlights}
                \item Coaching beginner and intermediate-level teams, planning and assisting at events
            \end{highlights}
        \end{onecolentry}

        \expspace

        \begin{twocolentry}{
            Apr 2012 – Sep 2016
        }
            \textbf{Student Dorm Activities}, Hans Dickmann Kolleg Karlsruhe
            
            \end{twocolentry}
                  \begin{onecolentry}
            \begin{highlights}
                \item Various roles in dorm self-government, planning and assisting at events
            \end{highlights}
        \end{onecolentry}
    

    \sectionspace
    \section{Skills}
    \sectspacetop
    \begin{onecolentry}
    %\textbf{Technologies}
            \begin{highlights}
                \item \textbf{Technologies:} C, C++, Python, C\#, Java, Git, GitLab, Confluence. Linux, Windows.\\
                 %Basic experience with GLSL, Vulkan, OpenGL, Blender, Katana, Unity, Docker, Jira.
                 Basic experience: Bash, Docker, Jira, GLSL, Vulkan, OpenGL, Blender, Katana, Unity.
                 \expspace
                %\end{highlights}
                %\expspace
                %\textbf{Languages}
                %\begin{highlights}
                \item \textbf{Languages:} German (native), English (proficient), French (basic)
            \end{highlights}
        \end{onecolentry}
    
%\newcommand{\vspacepubl}{\vspace{.2cm}}    
\newcommand{\vspacepubl}{\expspace}    
    \section{Publications}
    \sectspacetop
            \begin{samepage}
    \begin{onecolentry}
\textbf{Guiding Light Trees for Many-Light Direct Illumination.}\\
Eric Hamann, Alisa Jung, Carsten Dachsbacher\\
Eurographics 2023 – Short Papers
        \end{onecolentry}
\vspacepubl
    \begin{onecolentry}
    \textbf{Path Guiding with Vertex Triplet Distributions}\\
Vincent Schüßler, Johannes Hanika, Alisa Jung, Carsten Dachsbacher\\
Computer Graphics Forum 41(4), EGSR 2022
    \end{onecolentry}
\vspacepubl
    \begin{onecolentry}
    \textbf{Improving Spectral Upsampling with Fluorescenc}e\\
Lars König, Alisa Jung, Carsten Dachsbacher\\
MAM2020: Eurographics Workshop on Material Appearance Modeling
    \end{onecolentry}
\vspacepubl
    \begin{onecolentry}
\textbf{Detecting Bias in Monte Carlo Renderers using Welch’s t-test}\\
Alisa Jung, Johannes Hanika, Carsten Dachsbacher\\
Journal of Computer Graphics Techniques Vol. 9 (2), 2020. Presented at I3D 2021.    
    \end{onecolentry}
\vspacepubl
    \begin{onecolentry}
\textbf{Spectral Mollification for Bidirectional Fluorescence}\\
Alisa Jung, Johannes Hanika, Carsten Dachsbacher\\
Computer Graphics Forum 39(2) (Proceedings of Eurographics) 2020
    \end{onecolentry}
\vspacepubl
    \begin{onecolentry}
\textbf{Wide Gamut Spectral Upsampling with Fluorescence}\\
Alisa Jung, Alexander Wilkie, Johannes Hanika, Wenzel Jakob, Carsten Dachsbacher\\
Computer Graphics Forum 38(4), EGSR 2019, runner-up for best paper award
    \end{onecolentry}
\vspacepubl
    \begin{onecolentry}
    \textbf{Selective guided sampling with complete light transport paths}\\
Florian Reibold, Johannes Hanika, Alisa Jung, Carsten Dachsbacher\\
ACM Transactions on Graphics 37(6) (Proceedings of SIGGRAPH Asia 2018)
    \end{onecolentry}
\vspacepubl
    \begin{onecolentry}
    \textbf{A Simple Diffuse Fluorescent BBRRDF Model}\\
    Alisa Jung, Johannes Hanika, Steve Marschner, Carsten Dachsbacher\\
    MAM2018: Eurographics Workshop on Material Appearance Modeling
    \end{onecolentry}
        \end{samepage}

\end{document}